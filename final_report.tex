\documentclass[a4paper]{article}
\usepackage[T1]{fontenc}
\usepackage[english]{babel}
\usepackage{amsmath}
\usepackage{graphicx}
\usepackage{url}
\usepackage{verbatim}
\usepackage{titlesec}
\setcounter{secnumdepth}{5}   
\setcounter{tocdepth}{5} 

% Hypertext
\usepackage{hyperref}

%Bibliobraphy
\usepackage{natbib}
\usepackage{bibtopic}
\usepackage{url}

\usepackage[utf8]{inputenc} % Krävs för att svenska tecken ska läsas korrekt i vissa system.
%\usepackage[latin1]{inputenc} % Om svenska tecken inte fungerar korrekt, försök att byta ut föregående rad mot denna (eller testa utan någon av raderna)

%Allow the use of \verbatimtabinput which includes external files, and handling tabs correctly
\usepackage{moreverb}
\def\verbatimtabsize{4\relax} 

\bibliographystyle{unsrt}

%Remove red boxes due to the hyperref
\hypersetup{
    colorlinks,
    citecolor=black,
    filecolor=black,
    linkcolor=black,
    urlcolor=black
}
%%%%%%%%%%%%%%%%%%%%%%%%%%%%%

\title{ETS200 -- Group K\\--\\ Final Report \\
How can web applications be tested?
%TODO maybe add a better title
}

\author{Fernando de Andrade Pereira
\\Jesper Bonna
\\Sadat Tokhi
%TODO put your full names, guys
}

\begin{document}
\maketitle
\thispagestyle{empty}
\clearpage

\tableofcontents
\thispagestyle{empty}
\clearpage

\setcounter{page}{1}

\section{Strategies for testing web applications}
%TODO An explanation of current strategies for testing web applications.

\subsection{White box strategy}
%TODO

\subsection{Black box strategy}
%TODO

\subsection{Gray box strategy}
%TODO

\subsection{When is which strategy most applicable?}
%TODO An analysis of the strategies named above.

\section{Tools for testing web applications}
%TODO Common tools for testing web based applications. When is each tool most useful?

\section{Case studies}
%TODO Fernando: Which cases have been studied? Which technics have been applied? How can the results be evaluated?

\subsection{AJAX web applications \cite{mtr08}}
AJAX is a technology that can be used to develop Rich Internet Applications.
Since it use assynchronous messages between client and server and it allows dynamic page alteration in a different way than the multipage Web paradigm, it introduces new chalenges to the testing of these web applications.

\subsubsection{Techniques}
In the paper, four Web application testing techniques are introduced: \emph{model-based}, \emph{code-coverage-based}, \emph{black box} -- three tradicional Web applications testing techniques -- and \emph{state-based} -- this last one proposed as more adequate to handle with AJAX applications.

\paragraph{Model-based testing}

is a \emph{white box} testing technique.
A model of the application is constructed using reverse engineering and Web crawling technique -- a program that automatically transverses the Web's hyperlink structure, retrieving the content of the Web pages.
It generates a graph where the nodes represent pages and edges represent links.
The test cases will be defined chosing a coverage criteria and applying to this model. 

\paragraph{Code-coverage-based testing} 

is another \emph{white box} testing technique. 
It uses a control flow model of the application where nodes represent statements execute by the Web server or by the client and edges represent control transfer.
Theoretically, it can be use to test a AJAX Web application, but due to the complexity and dynamism of AJAX Web pages, this technique is very limited for this use.

\paragraph{Black box testing} 

uses functional requirements of the application to define test cases, i.e., scenarios that can be accomplished by a Web visitor through the browser.

\paragraph{State-based testing}

is proposed by the autors of the paper as a more efficient way of test a AJAX Web application.

Using AJAX, the state of the elements in the client-side can change dynamically.
A \emph{state} is considered a possible instance of the DOM in a Web page, i.e., it's characterized by the HTML elements.

A finite state machine describes the the behavior of the Web page, containing the states (equivalence classes can be used to reduce the total number of states) and the transitions, related to methods triggered by GUI events -- interations with the user, like button clicks -- and \emph{callback} functions -- tht happen when the page receive a response from ther server.

The method consists in: 

\begin{enumerate}
\item \emph{FSM construcion}. The FSM that models a AJAX Web page an be constructed during the design phase or can be reverse enginnered from the code.
\item \emph{Path extraction}. One coverage criteria (states or transitions coverage) is chosen and the paths are extracted from the FSM according to it.
\item \emph{Test cases generation}. Each path is converted into a test case.
\item \emph{Test cases execution}. A tool that works in HTTP request-response level is used execute the test. cases
\end{enumerate}

\subsubsection{Experiments}

The experiment consists in use the four testing techniques introduced to test two diffentent Web applications composed of Java, JSP and Javascript code -- \emph{Photoshare} and \emph{the Organizer}.

\paragraph{Preparation and procedure}



\paragraph{Results and conclusion}



\subsection{•}
%TODO find another article about testing

\section{Major problems with web-based testing today}
%TODO There is a lot of problems with web based testing today. Mostly it’s too time consuming and developers doesn’t have the time to test before release.

\section{What does the future hold?}
%TODO How can strategies and method

\newpage
\appendix

\begin{btSect}[alpha]{literature}
\section{References}
\btPrintAll
\end{btSect}

\end{document}
